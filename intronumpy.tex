\subsection{Vectores y matrices.} \index{Vectores! Definición en matlab} \index{Matrices! Definición en matlab} Una de las característica más interesantes de Matlab, es la posibilidad de crear fácilmente matrices. Se pueden crear de muchas maneras, la más elemental de todas ellas, emplea el operador de asignación $=$, y los símbolos especiales $[$, $]$, el punto y coma $;$ y la coma $,$. Las matrices se crean introduciendo los valores de sus elementos por filas, separados por comas o espacios. Una vez introducidos todos los elementos de una fila, se añade un punto y coma, o se pulsa la tecla \emph{intro}, y se añaden los elementos de la fila siguiente. El siguiente ejemplo muestra como crear una matriz de de dos filas y tres columnas:
\begin{verbatim}
 >> matriz23 =[ 1 3 4 ;3 5 -1]

matriz23 =

     1     3     4
     3     5    -1
\end{verbatim}
o también:
 \begin{verbatim}
  >> matriz23 =[ 1 3 4 
3 5 -1]

matriz23 =

     1     3     4
     3     5    -1
 \end{verbatim}

En el primer caso, se empleado el punto y coma para separar las filas y en el segundo se ha empleado la tecla \emph{intro}.  En ambos se emplea el símbolo $[$ para indicar a Matlab que queremos empezar a construir una matriz, y el símbolo $]$ para indicar a Matlab que hemos terminado de construirla. Una vez construida, Matlab nos devuelve en la ventana de comandos la Matriz completa. Matlab nos permite además emplear cada elemento de una matriz como si se tratase de una variable, es decir, se puede asignar  a los elementos de una matriz un valor numérico, el resultados de una operación o un valor guardado en otra variable:
\begin{verbatim}
>> a=1

a =

          1.00

>> b=2

b =

          2.00

>> mtr=[ a a+b a-b; 1 0.5 cos(0)]

mtr =

          1.00          3.00         -1.00
          1.00          0.50          1.00

>> 
\end{verbatim}

Matlab considera las matrices como la forma básica de sus variables, así para Matlab un escalar es una matriz de una fila por una columna. Un vector \emph{fila} de 3 elementos es una matriz de una fila por tres columnas y un vector \emph{columna} de tres elementos es una matriz de tres filas y una columna.

\paragraph*{Indexación.} \index{Indexación en matlab}Al igual que se hace en álgebra, Matlab es capaz de referirse a un elemento cualquiera de una matriz empleando índices para determinar su posición (fila y columna) dentro de la matriz.
\begin{equation*}
a=
\begin{pmatrix}
a_{11}&a_{12}&a_{13}\\
a_{21}&a_{22}&a_{23}\\
a_{31}&a_{32}&a_{33}
\end{pmatrix}
\end{equation*}

El criterio para referirse a un elemento concreto de una matriz, en Matlab es el mismo: se indica el nombre de la variable que contiene la matriz y a continuación, entre paréntesis y separados por una coma, el índice de su fila y después él de su columna:
 \begin{verbatim}
 >> a=[1 2 3; 4 5 6; 7 8 9]

a =

          1.00          2.00          3.00
          4.00          5.00          6.00
          7.00          8.00          9.00

>> a(1,2)

ans =

          2.00

>> a(2,1)

ans =

          4.00

>> 
 \end{verbatim}
 
Es interesante observar de nuevo cómo Matlab asigna por defecto el valor del elemento buscado a la variable \texttt{ans}. Como ya se ha dicho, es mejor asignar siempre una variable a los resultados, para asegurarnos de que no los perdemos al realizar nuevas operaciones:
\begin{verbatim}
>> a12=a(1,2)

a12 =

          2.00

>> a21=a(2,1)

a21 =

          4.00

>> 
\end{verbatim}
Ahora hemos creado dos variables nuevas que contienen los valores de los elementos $a_{12}$ y $a_{21}$ de la matriz $a$. 

Matlab puede seleccionar dentro de una matriz no solo elementos aislados, sino también submatrices completas. Para ello, emplea un símbolo reservado, el símbolo \emph{dos puntos} $:$. Este símbolo se emplea para recorrer valores desde un valor inicial hasta un valor final, con un incremento o paso fijo. La sintaxis es: \texttt{inicio:paso:fin}, por ejemplo podemos recorrer los números enteros de cero a 8 empleando un paso 2:
\begin{verbatim}
>> 0:2:8

ans =

             0          2.00          4.00          6.00          8.00

>> 
\end{verbatim}

El resultado nos da la lista de los números $0, 2, 4, 6, 8$.
Además, si no indicamos el tamaño del paso,  Matlab tomará por defecto un paso igual a uno. En este caso basta emplear un único símbolo \emph{dos puntos} para separar el valor de inicio del valor final:
 \begin{verbatim}
 >> 1:5

ans =

          1.00          2.00          3.00          4.00          5.00

>>
 \end{verbatim}
 
 Podemos emplear el símbolo \emph{dos puntos},\index{Indexación con el operador :} \index{": Operador de indexación} para obtener submatrices de una matriz dada. Así por ejemplo si construimos una matriz de cuatro filas por cinco columnas:
 \begin{verbatim}
 >> matriz=[1 2 4 5 6
3 5 -6 0 2
4 5 8 9 0
3 3 -1 2 0]

matriz =

          1.00          2.00          4.00          5.00          6.00
          3.00          5.00         -6.00          0.00          2.00
          4.00          5.00          8.00          9.00          0.00
          3.00          3.00         -1.00          2.00          0.00

>> 
 \end{verbatim}
 Podemos obtener el vector formado por los tres últimos elementos de su segunda fila:
 \begin{verbatim}
 >> fil=matriz(2,3:5)

fil =

         -6.00             0          2.00

>> 
 \end{verbatim}

 o la submatriz de tres filas por tres columnas formada por los elementos que ocupan las filas 2 a 4 y las columnas 3 a 5:
\begin{verbatim}
 >> subm=matriz(2:4,3:5)

 subm =

         -6.00          0                2.00
	          8.00          9.00             0  
         -1.00          2.00             0  

 >> 
\end{verbatim}

o el vector columna formado por su segunda columna completa:
\begin{verbatim}
>> matriz(1:4,2)

ans =

          2.00
          5.00
          5.00
          3.00

>> 
\end{verbatim} 

De hecho, si deseamos seleccionar todos los elementos en una fila o una columna, podemos emplear el símbolo \texttt{:} directamente sin indicar principio ni fin,

\begin{verbatim}

>> matriz(:,2)

ans =

          2.00
          5.00
          5.00
          3.00

>> matriz(3,:)
ans =

     4.00     5.00     8.00    9.00    0.00
\end{verbatim}
 
\label{index}A parte de la indexación típica del álgebra de los elementos de una matriz indicando su fila y columna, en Matlab es posible referirse a un  elemento de una matriz empleando un único índice. En este caso, Matlab cuenta los elementos por columnas, de arriba abajo y de izquierda a derecha, 

\begin{equation*}
A=
\begin{pmatrix}
a_1&a_4&a_7\\
a_2&a_5&a_8\\
a_3&a_6&a_9
\end{pmatrix}
\end{equation*}

Así por ejemplo, en una matriz $A$ de $3$ filas y $4$ columnas, 
\begin{verbatim}
>> A=[3 0 -1 0; 2 1 5 7; 1 3 9 8]
A =

     3     0    -1     0
     2     1     5     7
     1     3     9     8
\end{verbatim}
las expresiones,
\begin{verbatim}
>> A(2,3)

and =
     5
\end{verbatim}
y
\begin{verbatim}
>> A(8)

ans =
     5
\end{verbatim} 
hacen referencia al mismo elemento de la matriz $A$. 
